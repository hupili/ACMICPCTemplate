\subsection{优先队列优化的dijkstra}
	\begin{lstlisting}[language=c++]
#include<cstdio>
#include<cstring>
#include<iostream>
#include<algorithm>
#include<queue>
#include<vector>
using namespace std;
const int MAXN=100;
const int MAXM=1000;
int N,L;
int head[MAXN];
struct edges
{
	int to,next,cost;
} edge[MAXM];
int dist[MAXN];
class states
{
public:
	int cost,id;
};
class cmp
{
public:
	bool operator ()(const states &i,const states &j)
	{
		return i.cost>j.cost;
	}
};
void init(int n)
{
	N=n;
	L=0;
	for (int i=0; i<n; i++)
		head[i]=-1;
}
void add_edge(int x,int y,int cost)
{
	edge[L].to=y;
	edge[L].cost=cost;
	edge[L].next=head[x];
	head[x]=L++;
}
int dijkstra(int s,int t)
{
	memset(dist,63,sizeof(dist));
	states u;
	u.id=s;
	u.cost=0;
	dist[s]=0;
	priority_queue<states,vector<states>,cmp> q;
	q.push(u);
	while (!q.empty())
	{
		u=q.top();
		q.pop();
		if (u.id==t) return dist[t];
		if (u.cost!=dist[u.id]) continue;
		for (int i=head[u.id]; i!=-1; i=edge[i].next)
		{
			states v=u;
			v.id=edge[i].to;
			if (dist[v.id]>dist[u.id]+edge[i].cost)
			{
				v.cost=dist[v.id]=dist[u.id]+edge[i].cost;
				q.push(v);
			}
		}
	}
	return -1;
}
int main()
{
	int n,m;
	scanf("%d%d",&n,&m);
	init(n);
	for (int i=0; i<m; i++)
	{
		int x,y,z;
		scanf("%d%d%d",&x,&y,&z);
		add_edge(x,y,z);
		add_edge(y,x,z);
	}
	int s,t;
	scanf("%d%d",&s,&t);
	printf("%d\n",dijkstra(s,t));
	return 0;
}
	\end{lstlisting}