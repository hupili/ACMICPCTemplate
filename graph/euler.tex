\subsection{欧拉路}
	\subsubsection{有向图}
	\begin{lstlisting}[language=c++]
void solve(int x)
{
	int i;
	if (!match[x])
	{
		path[++l]=x;
		return ;
	}
	for (i=1; i<=n; i++)
		if (b[x][i])
		{
			b[x][i]--;
			match[x]--;
			solve(i);
		}
	path[++l]=x;
}
	\end{lstlisting}
	
	\subsubsection{无向图}
	\begin{lstlisting}[language=c++]
void solve(int x)
{
	int i;
	if (!match[x])
	{
		path[++l]=x;
		return ;
	}
	for (i=1; i<=n; i++)
		if (b[x][i])
		{
			b[x][i]--;
			b[i][x]--;
			match[x]--;
			match[i]--;
			solve(i);
		}
	path[++l]=x;
}
	\end{lstlisting}

	\subsubsection{混合图}
	zju1992
	\begin{lstlisting}[language=c++]
int in[MAXN+100],out[MAXN+100];
int main()
{
	int t;
	scanf("%d",&t);
	while (t--)
	{
		int n,m;
		scanf("%d%d",&n,&m);
		N=n+2;L=-1;
		for (int i=0;i<N;i++)
			head[i]=-1;
		memset(in,0,sizeof(in));
		memset(out,0,sizeof(out));

		for (int i=0;i<m;i++)
		{
			int x,y,z;
			scanf("%d%d%d",&x,&y,&z);
			in[y]++;out[x]++;
			if (!z)
				add_edge(x,y,1);
		}
		int flag=1;
		for (int i=1;i<=n;i++)
		{
			if (in[i]-out[i]>0)
				add_edge(i,n+1,(in[i]-out[i])/2);
			else
			if (out[i]-in[i]>0)
				add_edge(0,i,(out[i]-in[i])/2);
			//printf("%d %d %d\n",i,out[i],in[i]);
			if ((in[i]+out[i])&1)
			{
				flag=0;
				break;
			}
		}
		maxflow(0,n+1);
		for (int i=head[0];i!=-1;i=edge[i].next)
			if (edge[i].cap>0 && edge[i].cap>edge[i].flow)
			{
				flag=0;
				break;
			}
		if (flag)
			puts("possible");
		else
			puts("impossible");
	}
	return 0;
}
	\end{lstlisting}