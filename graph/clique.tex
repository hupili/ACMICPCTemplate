\subsection{最大团以及相关知识}
    \begin{description}
	\item[独立集:] 独立集是指图的顶点集的一个子集,该子集的导出子图不含边.如果一个独立集不是任何一个独立集的子集, 那么称这个独立集是一个极大独立集.一个图中包含顶点数目最多的独立集称为最大独立集。最大独立集 一定是极大独立集,但是极大独立集不一定是最大的独立集。
	\item[支配集:] 与独立集相对应的就是支配集,支配集也是图顶点集的一个子集,设$S$是图$G$的一个支配集,则对于图中的任意一个顶点$u$,要么属于集合$s$, 要么与$s$中的顶点相邻。在$s$中除去任何元素后$s$不再是支配集,则支配集$s$是极小支配集。称$G$的所有支配集中顶点个数最 少的支配集为最小支配集,最小支配集中的顶点个数成为支配数。
	\item[最小点的覆盖:] 最小点的覆盖也是图的顶点集的一个子集,如果我们选中一个点,则称这个点将以他为端点的所有边都覆盖了。将图中所有的边都覆盖所用顶点数最少,这个集合就是最小的点的覆盖。
	\item[最大团:] 图$G$的顶点的子集,设$D$是最大团,则$D$中任意两点相邻。若$u$,$v$是最大团,则$u$,$v$有边相连,其补图$u$,$v$没有边相连,所以图$G$的最大团$=$其补图的最大独立集。给定无向图$G=(V,E)$,如果$U$属于$V$,并且对于任意$u$,$v$包含于$U$ 有$<u,v>$包含于$E$,则称$U$是$G$的完全子图,$G$的完全子图$U$是$G$的团,当且仅当$U$不包含在$G$的更大的完全子图中,$G$的最大团是指$G$中所含顶点数目最多的团。如果$U$属于$V$,并且对于任意$u,v$包含于$U$有$<u,v>$不包含于$E$,则称$U$是$G$的空子图,$G$的空子图U是G的独立集,当且仅当$U$不包含在$G$的更大的独立集,$G$的最大团是指$G$中所含顶点数目最多的独立集。
	\item[一些性质:] 最大独立集$+$最小覆盖集$=V$,最大团$=$补图的最大独立集,最小覆盖集$=$最大匹配
    \end{description}
    
    \begin{lstlisting}[language=c++]
////////////////////////////////////////////////////////////////////////
//经典NPC:最大团问题
//maximum clique
//dist是两点的距离,需要初始化
//double和int看情况而定
//n是顶点数
//从1开始编号
#include <iostream>
#include <cstdio>
#include <cstring>
#include <cmath>
using namespace std;

int n,k;
struct graph
{
    int x,y;
}G[55];
int dist[55][55];
bool g[55][55];
int list[55][55],s[55],degree[55],behide[55];
int found,curmax,curobj;

int distanc(graph a,graph b)
{
    return ((a.x-b.x)*(a.x-b.x)+(a.y-b.y)*(a.y-b.y));
}

void sortdegree()
{
    for (int i = 1;i <= n;i++)
    {
        int k = i;
        for (int j = i+1;j <= n;j++)
            if (degree[j] < degree[k])
                k = j;
        if (k != i)
        {
            swap(degree[i],degree[k]);
            for (int l = 1;l <= n;l++)
                swap(g[i][l],g[k][l]);
            for (int l = 1;l <= n;l++)
                swap(g[l][i],g[l][k]);
        }
    }
}

void dfs(int d)
{
    if (d-1 > curmax)
    {
        found = 1;
        return;
    }
    for (int i = 1;i < list[d-1][0]-curmax+d;i++)
    if (!found && d+behide[list[d-1][i]+1] > curmax
        && (list[d-1][0] == i || d+behide[list[d-1][i+1]] > curmax))
        {
            list[d][0] = 0;
            for (int j = i+1;j <= list[d-1][0];j++)
            if (g[list[d-1][j]][list[d-1][i]])
                list[d][++list[d][0]] = list[d-1][j];
            if (list[d][0] == 0 || d+behide[list[d][1]] > curmax)
                dfs(d+1);
        }
}

void solve()
{
    sortdegree();
    behide[n+1] = 0;
    behide[n] = 1;
    for (int i = n-1;i > 0;i--)
    {
        curmax = behide[i+1];
        found = list[1][0] = 0;
        for (int j = i+1;j <= n;j++)
        if (g[j][i])
            list[1][++list[1][0]] = j;
        dfs(2);
        behide[i] = curmax+found;

    }
}

int check(int mindist)
{
    memset(g,false,sizeof(g));
    for (int i = 1;i < n;i++)
    for (int j = i+1;j <= n;j++)
        if (dist[i][j] >= mindist)//这个是约束条件~
            g[i][j] = g[j][i] = true;
    for (int i = 1;i <= n;i++)
    {
        degree[i] = 0;
        for (int j = 1;j <= n;j++)
            degree[i] += g[i][j];
    }
    solve();
    return behide[1];
}

int main()
{
    while (scanf("%d%d",&n,&k) != EOF)
    {
        for (int i = 1;i <= n;i++)
            scanf("%d%d",&G[i].x,&G[i].y);
        int l,r,mid;
        r = l = 0;
        for (int i = 1;i < n;i++)
        for (int j = i+1;j <= n;j++)
        {
            dist[i][j] = distanc(G[i],G[j]);
            if (dist[i][j] > r)
                r = dist[i][j];
            dist[j][i] = dist[i][j];
        }
        r++;
        while (l+1 < r)
        {
            mid = (l+r)/2;
            if (check(mid) >= k)
                l = mid;
            else
                r = mid;
        }
        printf("%.2lf\n",sqrt((double)l));
    }
}
    \end{lstlisting}
