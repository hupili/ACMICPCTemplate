\subsection{爬山法计算器}
	注意灵活运用。\\
	双目运算符在calc()中,左结合单目运算符在P()中,右结合单目运算符在calc\_exp中。(但是还没遇到过。。)\\
	\begin{lstlisting}[language=c++]
#include <iostream>
#include <cstdio>
#include <cstring>
#include <algorithm>
#include <string>
using namespace std;

char s[100000];
int n,cur;
const string OP = "+-*";

char next_char()
{
	if (cur >= n) return EOF;
	return s[cur];
}

int get_priority(char ch)
{
	if (ch == '*')  return 2;
	return 1;
}

int P();

int calc(int a,char op,int b)
{
	if (op == '+')
		return a+b;
	if (op == '-')
		return a-b;
	if (op == '*')
		return a*b;
}

int calc_exp(int p)
{
	int a = P();
	while ((OP.find(next_char()) != OP.npos) && (get_priority(next_char()) >= p))
	{
		char op = next_char();
		cur++;
		a = calc(a,op,calc_exp(get_priority(op)+1));
	}
	return a;
}

int totvar,m,var[26],varid[26];

int P()
{
	if (next_char() == '-')
	{
		cur++;
		return -P();
	}
	else if (next_char() == '+')
	{
		cur++;
		return P();
	}
	else if (next_char() == '(')
	{
		cur++;
		int res = calc_exp(0);
		cur++;
		return res;
	}
	else
	{
		cur++;
		//cout << "getvar at " << cur << ' ' << var[varid[s[cur]-'a']] << endl;
		return var[varid[s[cur-1]-'a']];
	}
}

int id[26],minid;

int main()
{
	while (true)
	{
		scanf("%d%d",&totvar,&var[0]);
		if (totvar == 0 && var[0] == 0)  break;
		for (int i = 1;i < totvar;i++)
			scanf("%d",&var[i]);
		scanf("%d",&m);
		scanf("%s",s);
		for (int i = 0;i < 26;i++)
			id[i] = -1;
		minid = 0;
		n = strlen(s);
		for (int i = 0;i < n;i++)
			if (s[i] >= 'a' && s[i] <= 'z')
			{
				if (id[s[i]-'a'] == -1)
				{
					id[s[i]-'a'] = minid;
					minid++;
				}
				s[i] = 'a'+id[s[i]-'a'];
			}
		for (int i = 0;i < totvar;i++)
			varid[i] = i;
		int res = 0;
		do
		{
			cur = 0;
			int tmp = calc_exp(0);
			if (tmp == m)
			{
				res++;
				break;
			}
		}
		while (next_permutation(varid,varid+totvar));
		//puts(s);
		if (res > 0)
			puts("YES");
		else
			puts("NO");
	}
	return 0;
}
	\end{lstlisting}
