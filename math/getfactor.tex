\subsection{分解质因数}
    \subsubsection{米勒拉宾+分解因数}
	\begin{lstlisting}[language=c++]
#include<ctime>
#include<iostream>
#define bint long long
using namespace std;
const int TIME = 8;//测试次数,8~10够了
int factor[100],fac_top = -1;

//计算两个数的gcd
bint gcd(bint small,bint big)
{
    while(small)
    {
        swap(small,big);
        small%=big;
    }
    return abs(big);
}

//ret = (a*b)%n (n<2^62)
bint muti_mod(bint a,bint b,bint n)
{
    bint exp = a%n, res = 0;
    while(b)
    {
        if(b&1)
        {
            res += exp;
            if(res>n) res -= n;
        }
        exp <<= 1;
        if (exp>n) exp -= n;
        b>>=1;
    }
    return res;
}

// ret = (a^b)%n
bint mod_exp(bint a,bint p,bint m)
{
    bint exp=a%m, res=1; //
    while(p>1)
    {
        if(p&1)
            res=muti_mod(res,exp,m);
        exp = muti_mod(exp,exp,m);
        p>>=1;
    }
    return muti_mod(res,exp,m);
}

//miller-rabin法测试素数, time 测试次数
bool miller_rabin(bint n, int times)
{
    if(n==2)return 1;
    if(n<2||!(n&1))return 0;
    bint a, u=n-1, x, y;
    int t=0;
    while(u%2==0)
    {
        t++;
        u/=2;
    }
    srand(time(0));
    for(int i=0; i<times; i++)
    {
        a = rand() % (n-1) + 1;
        x = mod_exp(a, u, n);
        for(int j=0; j<t; j++)
        {
            y = muti_mod(x, x, n);
            if ( y == 1 && x != 1 && x != n-1 )
                return false; //must not
            x = y;
        }
        if( y!=1) return false;
    }
    return true;
}

bint pollard_rho(bint n,int c)//找出一个因子
{
    bint x,y,d,i = 1,k = 2;
    srand(time(0));
    x = rand()%(n-1)+1;
    y = x;
    while(true)
    {
        i++;
        x = (muti_mod(x,x,n) + c) % n;
        d = gcd(y-x, n);
        if (1 < d && d < n) return d;
        if( y == x) return n;
        if(i == k)
        {
            y = x;
            k <<= 1;
        }
    }
}

void findFactor(bint n,int k)//二分找出所有质因子,存入factor
{
    if(n==1)return;
    if(miller_rabin(n, TIME))
    {
        factor[++fac_top] = n;
        return;
    }
    bint p = n;
    while(p >= n)
        p = pollard_rho(p,k--);//k值变化,防止死循环
    findFactor(p,k);
    findFactor(n/p,k);
}

int main()
{
    bint cs,n,min;
    cin>>cs;
    while (cs--)
    {
        cin>>n;
        fac_top = min = -1;
        if(miller_rabin(n,TIME)) cout<<"Prime"<<endl;
        else
        {
            findFactor(n,107);
            for(int i=0; i<=fac_top; i++)
            {
                if(min<0||factor[i]<min)
                    min = factor[i];
            }
            cout<<min<<endl;
        }
    }
    return 0;
}
	\end{lstlisting}
    
    \subsubsection{暴力版本}
	\begin{lstlisting}[language=c++]
int N;
int num[30],fac[30];
void getFactor(int x)
{
    N=0;
    memset(num,0,sizeof(num));
    for (int i=0; prime[i]*prime[i]<=x && i<L; i++)
    {
        if (x%prime[i]==0)
        {
            while (x%prime[i]==0)
            {
                x/=prime[i];
                num[N]++;
            }
            fac[N++]=prime[i];
        }
    }
    if (x>1)
    {
        num[N]=1;
        fac[N++]=x;
    }
}
	\end{lstlisting}