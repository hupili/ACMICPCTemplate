\subsection{其它公式}
    \subsubsection{正多面体顶点着色}
	正四面体:$N = \frac{(n^{4}+11\times n^{2})}{24}$\\
	正六面体:$N = \frac{(n^{8}+17\times n^{4}+6\times n^{2})}{24}$\\
	正八面体:$N = \frac{(n^{6}+3\times n^{4}+12\times n^{3}+8\times n^{2})}{24}$\\
	正十二面体:$N = \frac{(n^{20}+15\times n^{10}+20\times n^{8}+24\times n^{4})}{60}$\\
	正二十面体:$N = \frac{(n^{12}+15\times n^{6}+44\times n^{4})}{60}$\\
    \subsubsection{求和公式}
	$\sum{k} = \frac{n\times (n+1)}{2}$\\
	$\sum{2k-1} = n^{2}$\\
	$\sum{k^{2}} = \frac{n\times (n+1)\times (2n+1)}{6}$\\
	$\sum{(2k-1)^{2}} = \frac{n\times (4n^{2}-1)}{3}$\\
	$\sum{k^{3}} = (\frac{n\times (n+1)}{2})^{2}$\\
	$\sum{(2k-1)^{3}} = n^{2}\times (2n^{2}-1)$\\
	$\sum{k^{4}} = \frac{n\times (n+1)\times (2n+1)\times (3n^{2}+3n-1)}{30}$\\
	$\sum{k^{5}} = \frac{n^{2}\times (n+1)^{2}\times (2n^{2}+2n-1)}{12}$\\
	$\sum{k\times (k+1)} = \frac{n\times (n+1)\times (n+2)}{3}$\\
	$\sum{k\times (k+1)\times (k+2)} = \frac{n\times (n+1)\times (n+2)\times (n+3)}{4}$\\
	$\sum{k\times (k+1)\times (k+2)\times (k+3)} = \frac{n\times (n+1)\times (n+2)\times (n+3)\times (n+4)}{5}$\\
    \subsubsection{几何公式}
	球扇形:\\
	全面积:$T = \pi r(2h+r_0)$,$h$为球冠高,$r_0$为球冠底面半径\\
	体积:$V = \frac{2\pi r^{2}h}{3}$\\
    \subsubsection{小公式}
    Pick 公式:$A = E\times 0.5+I-1$($A$是多边形面积,$E$是边界上的整点,$I$是多边形内部的整点)\\
    \\
    海伦公式:$S = \sqrt{p(p-a)(p-b)(p-c)}$,其中$p = \frac{(a+b+c)}{2}$,$abc$为三角形的三条边长\\
    \\
    求$\binom{n}{k}$中素因子$P$的个数:\\
    \begin {enumerate}
	\item 把$n$转化为$P$进制,并记它每个位上的和为$S1$
	\item 把$n-k$,$k$做同样的处理,得到$S2$,$S3$
    \end{enumerate}
    则$\binom{n}{k}$中素因子$P$的个数:$\frac{S2+S3-S1}{P-1}$\\
    \\
    枚举长为$n$含$k$个$1$的$01$串:\\
    \begin{lstlisting}[language=c++]
for (int s = (1 << k)-1,u = 1 << n;s < u;)
{
    ...;
    int b = s & -s;
    s = (s+b)|(((s^(s+b))>>2)/b);
}
    \end{lstlisting}