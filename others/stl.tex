\subsection{C++\&STL常用函数}
    \subsubsection{lower\_bound/upper\_bound}
	不解释\\
	\begin{lstlisting}[language=c++]
iterator lower_bound(const key_type &key )
\\`返回一个迭代器,指向键值>= key的第一个元素。`
iterator upper_bound(const key_type &key )
\\`返回一个迭代器,指向键值> key的第一个元素。`

#include <iostream>
#include <algorithm>
#include <vector>
using namespace std;

int main () {
  int myints[] = {10,20,30,30,20,10,10,20};
  vector<int> v(myints,myints+8);           
  // 10 20 30 30 20 10 10 20
  vector<int>::iterator low,up;

  sort (v.begin(), v.end());                
  // 10 10 10 20 20 20 30 30

  low=lower_bound (v.begin(), v.end(), 20);           
  // 10 10 10 20 20 20 30 30
  //          ^
  up= upper_bound (v.begin(), v.end(), 20);           
  // 10 10 10 20 20 20 30 30
  //                   ^

  cout << "lower_bound at position " << int(low- v.begin()) << endl;
  cout << "upper_bound at position " << int(up - v.begin()) << endl;

  return 0;
}
	\end{lstlisting}
	Output:\\
	\begin{lstlisting}[language=c++]
lower_bound at position 3
upper_bound at position 6
	\end{lstlisting}
	
    \subsubsection{rotate}
	把数组后一半搬到前面\\
	\begin{lstlisting}[language=c++]
template <class ForwardIterator>
  void rotate ( ForwardIterator first, ForwardIterator middle,
                ForwardIterator last );
	\end{lstlisting}
	
    \subsubsection{nth\_element}
	\begin{lstlisting}[language=c++]
template <class RandomAccessIterator>
  void nth_element ( RandomAccessIterator first, RandomAccessIterator nth,
                     RandomAccessIterator last );

template <class RandomAccessIterator, class Comapre>
  void nth_element ( RandomAccessIterator first, RandomAccessIterator nth,
                     RandomAccessIterator last, Compare comp );
	\end{lstlisting}
	
    \subsubsection{bitset}
	取用\\
	\begin{lstlisting}[language=c++]
bitset<4> mybits;

mybits[1]=1;             // 0010
mybits[2]=mybits[1];     // 0110
	\end{lstlisting}
	翻转\\
	\begin{lstlisting}[language=c++]
bitset<4> mybits (string("0001"));

cout << mybits.flip(2) << endl;     // 0101
cout << mybits.flip() << endl;      // 1010
	\end{lstlisting}
	运算\\
	\begin{lstlisting}[language=c++]
bitset<4> first (string("1001"));
bitset<4> second (string("0011"));

cout << (first^=second) << endl;          // 1010 (XOR,assign)
cout << (first&=second) << endl;          // 0010 (AND,assign)
cout << (first|=second) << endl;          // 0011 (OR,assign)

cout << (first<<=2) << endl;              // 1100 (SHL,assign)
cout << (first>>=1) << endl;              // 0110 (SHR,assign)

cout << (~second) << endl;                // 1100 (NOT)
cout << (second<<1) << endl;              // 0110 (SHL)
cout << (second>>1) << endl;              // 0001 (SHR)

cout << (first==second) << endl;          // false (0110==0011)
cout << (first!=second) << endl;          // true  (0110!=0011)

cout << (first&second) << endl;           // 0010
cout << (first|second) << endl;           // 0111
cout << (first^second) << endl;           // 0101
	\end{lstlisting}
	
    \subsubsection{multimap}
	遍历\\
	\begin{lstlisting}[language=c++]
multimap<char,int> mymm;
multimap<char,int>::iterator it;
char c;

mymm.insert(pair<char,int>('x',50));
mymm.insert(pair<char,int>('y',100));
mymm.insert(pair<char,int>('y',150));
mymm.insert(pair<char,int>('y',200));
mymm.insert(pair<char,int>('z',250));
mymm.insert(pair<char,int>('z',300));

for (c='x'; c<='z'; c++)
{
  cout << "There are " << (int)mymm.count(c);
  cout << " elements with key " << c << ":";
  for (it=mymm.equal_range(c).first; it!=mymm.equal_range(c).second; ++it)
    cout << " " << (*it).second;
  cout << endl;
}
/*
Output:

There are 1 elements with key x: 50
There are 3 elements with key y: 100 150 200
There are 2 elements with key z: 250 300
*/
	\end{lstlisting}
	二分查找\\
	\begin{lstlisting}[language=c++]
multimap<char,int> mymultimap;
multimap<char,int>::iterator it,itlow,itup;

mymultimap.insert(pair<char,int>('a',10));
mymultimap.insert(pair<char,int>('b',121));
mymultimap.insert(pair<char,int>('c',1001));
mymultimap.insert(pair<char,int>('c',2002));
mymultimap.insert(pair<char,int>('d',11011));
mymultimap.insert(pair<char,int>('e',44));

itlow=mymultimap.lower_bound ('b');  // itlow points to b
itup=mymultimap.upper_bound ('d');   // itup points to e (not d)

// print range [itlow,itup):
for ( it=itlow ; it != itup; it++ )
  cout << (*it).first << " => " << (*it).second << endl;
  
/*
Output:

b => 121
c => 1001
c => 2002
d => 11011
*/
	\end{lstlisting}
	删除\\
	\begin{lstlisting}[language=c++]
multimap<char,int> mymultimap;
multimap<char,int>::iterator it;

// insert some values:
mymultimap.insert(pair<char,int>('a',10));
mymultimap.insert(pair<char,int>('b',20));
mymultimap.insert(pair<char,int>('b',30));
mymultimap.insert(pair<char,int>('c',40));
mymultimap.insert(pair<char,int>('d',50));
mymultimap.insert(pair<char,int>('d',60));
mymultimap.insert(pair<char,int>('e',70));
mymultimap.insert(pair<char,int>('f',80));

it=mymultimap.find('b');
mymultimap.erase (it);                     
// erasing by iterator (1 element)

mymultimap.erase ('d');                    
// erasing by key (2 elements)

it=mymultimap.find ('e');
mymultimap.erase ( it, mymultimap.end() ); 
// erasing by range

// show content:
for ( it=mymultimap.begin() ; it != mymultimap.end(); it++ )
  cout << (*it).first << " => " << (*it).second << endl;
    
/*
Output:

a => 10
b => 30
c => 40
*/
	\end{lstlisting}