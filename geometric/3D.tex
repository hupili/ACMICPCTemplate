\subsection{三维几何}
	\subsubsection{Point定义}
		\begin{lstlisting}[language=c++]
struct Point3D
{
	double x,y,z;
	Point3D() {}
	Point3D(double _x,double _y,double _z)
	{
		x = _x;
		y = _y;
		z = _z;
	}
	Point3D operator -(const Point3D& b)const
	{
		return Point3D(x-b.x,y-b.y,z-b.z);
	}
	Point3D operator *(const Point3D& b)const
	{
		return Point3D(y*b.z-z*b.y,z*b.x-x*b.z,x*b.y-y*b.x);
	}
	double operator &(const Point3D& b)const
	{
		return x*b.x+y*b.y+z*b.z;
	}
};
//模
double Norm(Point3D p)
{
	return sqrt(p&p);
}
//`绕单位向量$V$旋转$\theta$角度`
Point3D Trans(Point3D pa,Point3D V,double theta)
{
	double s = sin(theta);
	double c = cos(theta);
	double x,y,z;
	x = V.x;
	y = V.y;
	z = V.z;
	Point3D pp = 
	Point3D(
		(x*x*(1-c)+c)*pa.x+(x*y*(1-c)-z*s)*pa.y+(x*z*(1-c)+y*s)*pa.z,
		(y*x*(1-c)+z*s)*pa.x+(y*y*(1-c)+c)*pa.y+(y*z*(1-c)-x*s)*pa.z,
		(x*z*(1-c)-y*s)*pa.x+(y*z*(1-c)+x*s)*pa.y+(z*z*(1-c)+c)*pa.z);
	return pp;
}
		\end{lstlisting}
	
	\subsubsection{经度纬度转换}
		直角坐标系与极坐标系转换:\\
		\[\begin{cases}
		x = r\times sin\theta \times cos\varphi \\
		y = r\times sin\theta \times sin\varphi \\
		z = r\times cos\theta\\
		\end{cases}
		\begin{cases}
		r = \sqrt{x\times 2 + y\times 2 + z\times 2}\\
		\varphi = arctan(\frac{y}{x})\\
		\theta = arccos(\frac{z}{r})\\
		\end{cases}r\in [0,+\infty),\varphi \in [0,2\pi],\theta \in [0,\pi] \]
		经度维度转换($lat1\in (-\frac{\pi}{2},\frac{\pi}{2}), lng1\in (-\pi,\pi)$)
		\begin{lstlisting}[language=c++]
Point3D getPoint3D(double lat,double lng,double r)
{
	lat += pi/2;
	lng += pi;
	return 
		Point3D(r*sin(lat)*cos(lng),r*sin(lat)*sin(lng),r*cos(lat));
}
		\end{lstlisting}
		
  
	\subsubsection{判断:直线相交}
		\begin{lstlisting}[language=c++]
bool LineIntersect(Line3D L1, Line3D L2) 
{ 
	Point3D s = L1.s-L1.e; 
	Point3D e = L2.s-L2.e; 
	Point3D p  = s*e; 
	if (ZERO(p)) return false;	  //是否平行 
	p = (L2.s-L1.e)*(L1.s-L1.e); 
	return ZERO(p&L2.e);		 //是否共面 
} 
		\end{lstlisting}
		
	\subsubsection{判断:线段相交}
	需要先判断是否在一个平面上:
	\begin{lstlisting}[language=c++]
bool inter(Point a,Point b,Point c,Point d)
{
	Point ret = (a-b)*(c-d);
	Point t1 = (b-a)*(c-a);
	Point t2 = (b-a)*(d-a);
	Point t3 = (d-c)*(a-c);
	Point t4 = (d-c)*(b-c);
	return sgn(t1&ret)*sgn(t2&ret) < 0 &&
				 sgn(t3&ret)*sgn(t4&ret) < 0;
}
	\end{lstlisting}
	
	\subsubsection{判断:三维向量是否为0}
		\begin{lstlisting}[language=c++]
inline bool ZERO(Point3D p)
{ 
	return (ZERO(p.x) && ZERO(p.y) && ZERO(p.z)); 
} 
		\end{lstlisting}
				
	\subsubsection{判断:点在直线上}
		\begin{lstlisting}[language=c++]
bool OnLine(Point3D p, Line3D L)
{ 
	return ZERO((p-L.s)*(L.e-L.s)); 
} 
		\end{lstlisting}
		
	\subsubsection{判断:点在线段上}
		\begin{lstlisting}[language=c++]
bool OnSeg(Point3D p, Line3D L)
{ 
	return (ZERO((L.s-p)*(L.e-p)) && 
		EQ(Norm(p-L.s)+Norm(p-L.e),Norm(L.e-L.s))); 
} 
		\end{lstlisting}
		
	\subsubsection{距离:点到直线}
		\begin{lstlisting}[language=c++]
double Distance(Point3D p, Line3D L)
{ 
	return (Norm((p-L.s)*(L.e-L.s))/Norm(L.e-L.s)); 
} 
		\end{lstlisting}

	\subsubsection{夹角}
		返回值是$[0,\pi]$之间的弧度
		\begin{lstlisting}[language=c++]
double Inclination(Line3D L1, Line3D L2) 
{ 
	Point3D u = L1.e - L1.s; 
	Point3D v = L2.e - L2.s; 
	return acos( (u & v) / (Norm(u)*Norm(v)) ); 
} 
		\end{lstlisting}