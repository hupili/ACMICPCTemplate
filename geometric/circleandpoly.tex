\subsection{一个圆与多边形面积交}
	\begin{lstlisting}[language=c++]
bool InCircle(Point a,double r)
{
	return cmp(a.x*a.x+a.y*a.y,r*r) <= 0; //`这里判断的时候EPS一定不要太小!!`
}

double CalcArea(Point a,Point b,double r)
{
	Point p[4];
	int tot = 0;
	p[tot++] = a;

	Point tv = Point(a,b);
	Line tmp = Line(Point(0,0),Point(tv.y,-tv.x));
	Point near = LineToLine(Line(a,b),tmp);
	if (cmp(near.x*near.x+near.y*near.y,r*r) <= 0)
	{
		double A,B,C;
		A = near.x*near.x+near.y*near.y;
		C = r;
		B = C*C-A;
		double tvl = tv.x*tv.x+tv.y*tv.y;
		double tmp = sqrt(B/tvl); //这样做只用一次开根
		p[tot] = Point(near.x+tmp*tv.x,near.y+tmp*tv.y);
		if (OnSeg(Line(a,b),p[tot]) == true)	tot++;
		p[tot] = Point(near.x-tmp*tv.x,near.y-tmp*tv.y);
		if (OnSeg(Line(a,b),p[tot]) == true)	tot++;
	}
	if (tot == 3)
	{
		if (cmp(Point(p[0],p[1]).Length(),Point(p[0],p[2]).Length()) > 0)
			swap(p[1],p[2]);
	}
	p[tot++] = b;

	double res = 0.0,theta,a0,a1,sgn;
	for (int i = 0;i < tot-1;i++)
	{
		if (InCircle(p[i],r) == true && InCircle(p[i+1],r) == true)
		{
			res += 0.5*xmult(p[i],p[i+1]);
		}
		else
		{
			a0 = atan2(p[i+1].y,p[i+1].x);
			a1 = atan2(p[i].y,p[i].x);
			if (a0 < a1)	a0 += 2*pi;
			theta = a0-a1;
			if (cmp(theta,pi) >= 0) theta = 2*pi-theta;
			sgn = xmult(p[i],p[i+1])/2.0;
			if (cmp(sgn,0) < 0) theta = -theta;
			res += 0.5*r*r*theta;
		}
	}
	return res;
}
	\end{lstlisting}
	调用
	\begin{lstlisting}[language=c++]
area2 = 0.0;
for (int i = 0;i < resn;i++) //遍历每条边,按照逆时针
	area2 += CalcArea(p[i],p[(i+1)%resn],r);
	\end{lstlisting}