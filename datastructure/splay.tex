\subsection{*Splay}
	持续学习中。\\
	注意节点的size值不一定是真实的值!如果有需要需要特别维护!\\
	\begin{enumerate}
	  \item 旋转和Splay操作
	  \item rank操作
	  \item insert操作(。。很多题目都有)
	  \item del操作(郁闷的出纳员)
	  \item 由数组建立Splay
	  \item 前驱后继(营业额统计)
	  \item Pushdown Pushup的位置
	  \item *。。。暂时想不起了
	\end{enumerate}
	节点定义。。\\
	\begin{lstlisting}[language=c++]
const int MaxN = 50003;

struct Node
{
	int size,key;

	Node *c[2];
	Node *p;
} mem[MaxN], *cur, *nil;
	\end{lstlisting}
	无内存池的几个初始化函数。\\
	\begin{lstlisting}[language=c++]
Node *newNode(int v, Node *p)
{
	cur->c[0] = cur->c[1] = nil, cur->p = p;
	cur->size = 1;
	cur->key = v;
	return cur++;
}

void Init()
{
	cur = mem;
	nil = newNode(0, cur);
	nil->size = 0;
}
	\end{lstlisting}
	带内存池的几个函数。\\
	\begin{lstlisting}[language=c++]
int emp[MaxN], totemp;

Node *newNode(int v, Node *p)
{
	cur = mem + emp[--totemp];
	cur->c[0] = cur->c[1] = nil, cur->p = p;
	cur->size = 1;
	cur->key = v;
	return cur;
}

void Init()
{
	for (int i = 0; i < MaxN; ++i)
		emp[i] = i;
	totemp = MaxN;
	cur = mem + emp[--totemp];
	nil = newNode(0, cur);
	nil->size = 0;
}

void Recycle(Node *p)
{
	if (p == nil)   return;
	Recycle(p->c[0]), Recycle(p->c[1]);
	emp[totemp++] = p - mem;
}
	\end{lstlisting}
	基本的Splay框架。维护序列用。\\
	一切下标从0开始。\\
	\begin{lstlisting}[language=c++]
struct SplayTree
{
	Node *root;
	void Init()
	{
		root = nil;
	}
	void Pushup(Node *x)
	{
		if (x == nil)   return;
		Pushdown(x); Pushdown(x->c[0]); Pushdown(x->c[1]);
		x->size = x->c[0]->size + x->c[1]->size + 1;
	}
	void Pushdown(Node *x)
	{
		if (x == nil)   return;
		//do something
	}
	void Rotate(Node *x, int f)
	{
		if (x == nil)   return;
		Node *y = x->p;
		y->c[f ^ 1] = x->c[f], x->p = y->p;
		if (x->c[f] != nil)
			x->c[f]->p = y;
		if (y->p != nil)
			y->p->c[y->p->c[1] == y] = x;
		x->c[f] = y, y->p = x;
		Pushup(y);
	}
	void Splay(Node *x, Node *f)
	{
		while (x->p != f)
		{
			Node *y = x->p;
			if (y->p == f)
				Rotate(x, x == y->c[0]);
			else
			{
				int fd = y->p->c[0] == y;
				if (y->c[fd] == x)
					Rotate(x, fd ^ 1), Rotate(x, fd);
				else
					Rotate(y, fd), Rotate(x, fd);
			}
		}
		Pushup(x);
		if (f == nil)
			root = x;
	}
	void Select(int k, Node *f)
	{
		Node *x = root;
		Pushdown(x);
		int tmp;
		while ((tmp = x->c[0]->size) != k)
		{
			if (k < tmp)	x = x->c[0];
			else
				x = x->c[1], k -= tmp + 1;
			Pushdown(x);
		}
		Splay(x, f);
	}
	void Select(int l, int r)
	{
		Select(l, nil), Select(r + 2, root);
	}
	Node *Make_tree(int a[], int l, int r, Node *p)
	{
		if (l > r)  return nil;
		int mid = l + r >> 1;
		Node *x = newNode(a[mid], p);
		x->c[0] = Make_tree(a, l, mid - 1, x);
		x->c[1] = Make_tree(a, mid + 1, r, x);
		Pushup(x);
		return x;
	}
	void Insert(int pos, int a[], int n)
	{
		Select(pos, nil), Select(pos + 1, root);
		root->c[1]->c[0] = Make_tree(a, 0, n - 1, root->c[1]);
		Splay(root->c[1]->c[0], nil);
	}
	void Insert(int v)
	{
		Node *x = root, *y = nil;
		while (x != nil)
		{
			y = x;
			y->size++;
			x = x->c[v >= x->key];
		}
		y->c[v >= y->key] = x = newNode(v, y);
		Splay(x, nil);
	}
	void Remove(int l, int r)
	{
		Select(l, r);
		//Recycle(root->c[1]->c[0]);
		root->c[1]->c[0] = nil;
		Splay(root->c[1], nil);
	}
};
	\end{lstlisting}
	例题:旋转区间赋值求和求最大子序列。\\
	注意打上懒标记后立即Pushup。Pushup(root-c[1]-c[0]),Pushup(root-c[1]),Pushup(root);\\
	\begin{lstlisting}[language=c++]
	void Pushup(Node *x)
	{
		if (x == nil)	return;
		Pushdown(x); Pushdown(x->c[0]); Pushdown(x->c[1]);
		x->size = x->c[0]->size+x->c[1]->size+1;

		x->sum = x->c[0]->sum+x->c[1]->sum+x->key;
		x->lsum = max(x->c[0]->lsum,
			x->c[0]->sum+x->key+max(0,x->c[1]->lsum));
		x->rsum = max(x->c[1]->rsum,
			x->c[1]->sum+x->key+max(0,x->c[0]->rsum));
		x->maxsum = max(max(x->c[0]->maxsum,x->c[1]->maxsum),
			x->key+max(0,x->c[0]->rsum)+max(0,x->c[1]->lsum));
	}
	void Pushdown(Node *x)
	{
		if (x == nil)	return;
		if (x->rev)
		{
			x->rev = 0;
			x->c[0]->rev ^= 1;
			x->c[1]->rev ^= 1;
			swap(x->c[0],x->c[1]);
			
			swap(x->lsum,x->rsum);
		}
		if (x->same)
		{
			x->same = false;
			x->key = x->lazy;
			x->sum = x->key*x->size;
			x->lsum = x->rsum = x->maxsum = max(x->key,x->sum);
			x->c[0]->same = true, x->c[0]->lazy = x->key;
			x->c[1]->same = true, x->c[1]->lazy = x->key;
		}
	}

int main()
{
	int totcas;
	scanf("%d",&totcas);
	for (int cas = 1;cas <= totcas;cas++)
	{
		Init();
		sp.Init();
		nil->lsum = nil->rsum = nil->maxsum = -Inf;
		sp.Insert(0);
		sp.Insert(0);

		int n,m;
		scanf("%d%d",&n,&m);
		for (int i = 0;i < n;i++)
			scanf("%d",&a[i]);
		sp.Insert(0,a,n);

		for (int i = 0;i < m;i++)
		{
			int pos,tot,c;
			scanf("%s",buf);
			if (strcmp(buf,"MAKE-SAME") == 0)
			{
				scanf("%d%d%d",&pos,&tot,&c);
				sp.Select(pos-1,pos+tot-2);
				sp.root->c[1]->c[0]->same = true;
				sp.root->c[1]->c[0]->lazy = c;
				sp.Pushup(sp.root->c[1]), sp.Pushup(sp.root);
			}
			else if (strcmp(buf,"INSERT") == 0)
			{
				scanf("%d%d",&pos,&tot);
				for (int i = 0;i < tot;i++)
					scanf("%d",&a[i]);
				sp.Insert(pos,a,tot);
			}
			else if (strcmp(buf,"DELETE") == 0)
			{
				scanf("%d%d",&pos,&tot);
				sp.Remove(pos-1,pos+tot-2);
			}
			else if (strcmp(buf,"REVERSE") == 0)
			{
				scanf("%d%d",&pos,&tot);
				sp.Select(pos-1,pos+tot-2);
				sp.root->c[1]->c[0]->rev ^= 1;
				sp.Pushup(sp.root->c[1]), sp.Pushup(sp.root);
			}
			else if (strcmp(buf,"GET-SUM") == 0)
			{
				scanf("%d%d",&pos,&tot);
				sp.Select(pos-1,pos+tot-2);
				printf("%d\n",sp.root->c[1]->c[0]->sum);
			}
			else if (strcmp(buf,"MAX-SUM") == 0)
			{
				sp.Select(0,sp.root->size-3);
				printf("%d\n",sp.root->c[1]->c[0]->maxsum);
			}
		}
	}
	return 0;
}
	\end{lstlisting}
	维护多个序列的时候,不需要建立很多Splay。只需要记录某个点在内存池中的绝对位置就可以了。\\
	需要操作它所在的序列时直接Splay到nil。此时Splay的root所在的Splay就是这个序列了。\\
	新建序列的时候需要多加入两个额外节点。如果某个Splay只有两个节点了需要及时回收。\\
	例题:Box(维护括号序列)\\
	\begin{lstlisting}[language=c++]
	\\`下面都是专用函数`
	\\`判断x在不在f里面`
	bool Ancestor(Node *x,Node *f)
	{
		if (x == f) return true;
		while (x->p != nil)
		{
			if (x->p == f)  return true;
			x = x->p;
		}
		return false;
	}
	\\`把Splay v插入到pos后面,pos=nil时新开一个序列`
	void Insert(Node *pos, Node *v)
	{
		int pl;
		if (pos == nil)
		{
			Init();
			Insert(0), Insert(0);
			pl = 0;
		}
		else
		{
			Splay(pos, nil);
			pl = root->c[0]->size;
		}
		Select(pl, nil), Select(pl + 1, root);
		root->c[1]->c[0] = v;
		v->p = root->c[1];
		Splay(v, nil);
	}
	\\`把[l,r]转出来(这里记录的是绝对位置)`
	void Select(Node *l, Node *r)
	{
		Splay(l, nil);
		int pl = root->c[0]->size - 1;
		Splay(r, nil);
		int pr = root->c[0]->size - 1;
		Select(pl, pr);
	}
	\\`分离[l,r]`
	Node *Split(Node *l, Node *r)
	{
		Select(l, r);
		Node *res = root->c[1]->c[0];
		root->c[1]->c[0] = res->p = nil;
		Splay(root->c[1], nil);
		if (root->size == 2)
		{
			Recycle(root);
			Init();
		}
		return res;
	}

int main(int argc, char const *argv[])
{
	freopen("P.in", "r", stdin);
	bool first = true;
	while (scanf("%d", &n) != EOF)
	{
		if (!first) puts("");
		first = false;
		Init();
		for (int i = 0; i < n; i++)
		{
			\\`建立独立的N个区间,记录绝对位置`
			sp.Init();
			sp.Insert(0), sp.Insert(0);
			sp.Insert(0,i+1),sp.Insert(1,i+1);
			sp.Select(0, 0), l[i] = sp.root->c[1]->c[0];
			sp.Select(1, 1), r[i] = sp.root->c[1]->c[0];
		}
		for (int i = 0; i < n; i++)
		{
			int f;
			scanf("%d", &f);
			if (f != 0)
			{
				\\`把[l[i],r[i]]插入到l[f-1]后面`
				Node *pos = sp.Split(l[i], r[i]);
				sp.Insert(l[f - 1], pos);
			}
		}
		scanf("%d", &n);
		for (int i = 0; i < n; i++)
		{
			scanf("%s", com);
			if (com[0] == 'Q')
			{
				int pos;
				scanf("%d", &pos);
				\\`求[l[pos-1],r[pos-1]]在哪个序列里面`
				sp.Splay(l[pos - 1], nil);
				sp.Select(1, nil);
				printf("%d\n", sp.root->key);
			}
			else
			{
				int u, v;
				scanf("%d%d", &u, &v);
				if (v == 0)
					sp.Insert(nil, sp.Split(l[u-1], r[u-1]));
				else
				{
					sp.Select(l[u-1],r[u-1]);
					if (sp.Ancestor(l[v-1],sp.root->c[1]->c[0]) == false)
						sp.Insert(l[v - 1], sp.Split(l[u-1], r[u-1]));
				}
			}
		}
	}
	return 0;
}
	\end{lstlisting} 